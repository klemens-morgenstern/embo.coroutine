% This text is proprietary.
% It's a part of presentation made by myself.
% It may not used commercial.
% The noncommercial use such as private and study is 

\documentclass{beamer}
\usepackage{xcolor}
\usepackage{tikz}
\usepackage{aeguill}
\usepackage{adjustbox}
\usepackage{listings}

\usetheme{Warsaw}
%\usecolortheme{dolphin}

\begin{document}

\lstset{language=C++,
                basicstyle=\ttfamily,
                keywordstyle=\color{blue}\ttfamily,
                stringstyle=\color{red}\ttfamily,
                commentstyle=\color{green}\ttfamily,
              %  columns=fixed,
                morecomment=[l][\color{magenta}]{\#}
}

\lstdefinelanguage{ASM}{
    morekeywords={str, add, move, sub, ldr, bx, mov,
    stmfd, nop, ldmfd bl},
    sensitive=false, % keywords are not case-sensitive
    morecomment=[l]{//}, % l is for line comment
    morecomment=[s]{/*}{*/}, % s is for start and end delimiter
    morestring=[b]" % defines that strings are enclosed in double quotes
} % 

\title{Developing high-performance Coroutines for ARMs}
\author{Klemens Morgenstern}
\date{10.03.2018}

\frame{\titlepage}



\begin{frame}
\frametitle{Introduction}
\begin{itemize}
\item Klemens Morgenstern
\item Electrical engineer
\item boost contributor
\item Open for contraction \& consulting
\item klemens.d.morgenstern@gmail.com
\end{itemize}
\end{frame}

\begin{frame}
\frametitle{Threading \& Coroutines}


\begin{columns}
\begin{column}{0.5\textwidth}
Threads
\begin{itemize}
\item<1-> Provide execution context
\item<2-> Scheduler interrupts 
\item<3-> Undeterministic order
\end{itemize}
\end{column}
\begin{column}{0.5\textwidth}
Coroutines
\begin{itemize}
\item<1-> Provide execution context
\item<2-> Deterministic exit points
\item<3-> Guaranteed execution order
\end{itemize}
\end{column}
\end{columns}

\end{frame}
\begin{frame}
\frametitle{Multithreading}
\begin{adjustbox}{max totalsize={.9\textwidth}{.7\textheight},center}
% generated by Plantuml 1.2018.00      
\definecolor{plantucolor0000}{RGB}{255,255,255}
\definecolor{plantucolor0001}{RGB}{74,100,132}
\definecolor{plantucolor0002}{RGB}{145,198,255}
\definecolor{plantucolor0003}{RGB}{0,0,0}
\begin{tikzpicture}[yscale=-1,pstyle0/.style={color=plantucolor0001,fill=white,line width=1.0pt}
,pstyle1/.style={color=plantucolor0001,line width=1.0pt,dash pattern=on 5.0pt off 5.0pt}
,pstyle2/.style={color=black,fill=plantucolor0002,line width=1.5pt}
,pstyle3/.style={color=plantucolor0001,fill=plantucolor0001,line width=1.0pt}
,pstyle4/.style={color=plantucolor0001,line width=1.0pt}
]
\draw[pstyle0] (210.9683pt,71.7461pt) rectangle (220.9683pt,102.2246pt);
\draw[pstyle0] (210.9683pt,254.6172pt) rectangle (220.9683pt,285.0957pt);
\draw[pstyle0] (322.2774pt,163.1816pt) rectangle (332.2774pt,193.6602pt);
\draw[pstyle1] (50pt,39.7461pt) -- (50pt,303.5742pt);
\draw[pstyle1] (215.4066pt,39.7461pt) -- (215.4066pt,303.5742pt);
\draw[pstyle1] (326.7157pt,39.7461pt) -- (326.7157pt,303.5742pt);
\draw[pstyle2] (8pt,3pt) rectangle (88.1286pt,34.7461pt);
\node at (15pt,10pt)[below right,color=black]{Scheduler};
\draw[pstyle2] (8pt,302.5742pt) rectangle (88.1286pt,334.3203pt);
\node at (15pt,309.5742pt)[below right,color=black]{Scheduler};
\draw[pstyle2] (179.4066pt,3pt) rectangle (248.53pt,34.7461pt);
\node at (186.4066pt,10pt)[below right,color=black]{Thread1};
\draw[pstyle2] (179.4066pt,302.5742pt) rectangle (248.53pt,334.3203pt);
\node at (186.4066pt,309.5742pt)[below right,color=black]{Thread1};
\draw[pstyle2] (290.7157pt,3pt) rectangle (359.8391pt,34.7461pt);
\node at (297.7157pt,10pt)[below right,color=black]{Thread2};
\draw[pstyle2] (290.7157pt,302.5742pt) rectangle (359.8391pt,334.3203pt);
\node at (297.7157pt,309.5742pt)[below right,color=black]{Thread2};
\draw[pstyle0] (210.9683pt,71.7461pt) rectangle (220.9683pt,102.2246pt);
\draw[pstyle0] (210.9683pt,254.6172pt) rectangle (220.9683pt,285.0957pt);
\draw[pstyle0] (322.2774pt,163.1816pt) rectangle (332.2774pt,193.6602pt);
\draw[pstyle3] (198.9683pt,67.7461pt) -- (208.9683pt,71.7461pt) -- (198.9683pt,75.7461pt) -- (202.9683pt,71.7461pt) -- cycle;
\draw[pstyle4] (50.0643pt,71.7461pt) -- (204.9683pt,71.7461pt);
\node at (57.0643pt,53.7461pt)[below right,color=black]{start (context switch)};
\draw[pstyle3] (203.9683pt,98.2246pt) -- (213.9683pt,102.2246pt) -- (203.9683pt,106.2246pt) -- (207.9683pt,102.2246pt) -- cycle;
\draw[pstyle4] (50.0643pt,102.2246pt) -- (209.9683pt,102.2246pt);
\node at (57.0643pt,84.2246pt)[below right,color=black]{interrupt};
\draw[pstyle3] (315.2774pt,128.7031pt) -- (325.2774pt,132.7031pt) -- (315.2774pt,136.7031pt) -- (319.2774pt,132.7031pt) -- cycle;
\draw[pstyle4] (215.9683pt,132.7031pt) -- (321.2774pt,132.7031pt);
\node at (222.9683pt,114.7031pt)[below right,color=black]{context switch};
\draw[pstyle3] (310.2774pt,159.1816pt) -- (320.2774pt,163.1816pt) -- (310.2774pt,167.1816pt) -- (314.2774pt,163.1816pt) -- cycle;
\draw[pstyle4] (50.0643pt,163.1816pt) -- (316.2774pt,163.1816pt);
\node at (57.0643pt,145.1816pt)[below right,color=black]{start (context switch)};
\draw[pstyle3] (315.2774pt,189.6602pt) -- (325.2774pt,193.6602pt) -- (315.2774pt,197.6602pt) -- (319.2774pt,193.6602pt) -- cycle;
\draw[pstyle4] (50.0643pt,193.6602pt) -- (321.2774pt,193.6602pt);
\node at (57.0643pt,175.6602pt)[below right,color=black]{interrupt};
\draw[pstyle3] (226.9683pt,220.1387pt) -- (216.9683pt,224.1387pt) -- (226.9683pt,228.1387pt) -- (222.9683pt,224.1387pt) -- cycle;
\draw[pstyle4] (220.9683pt,224.1387pt) -- (326.2774pt,224.1387pt);
\node at (232.9683pt,206.1387pt)[below right,color=black]{context switch};
\draw[pstyle3] (198.9683pt,250.6172pt) -- (208.9683pt,254.6172pt) -- (198.9683pt,258.6172pt) -- (202.9683pt,254.6172pt) -- cycle;
\draw[pstyle4] (50.0643pt,254.6172pt) -- (204.9683pt,254.6172pt);
\node at (57.0643pt,236.6172pt)[below right,color=black]{reenter (context switch)};
\draw[pstyle3] (203.9683pt,281.0957pt) -- (213.9683pt,285.0957pt) -- (203.9683pt,289.0957pt) -- (207.9683pt,285.0957pt) -- cycle;
\draw[pstyle4] (50.0643pt,285.0957pt) -- (209.9683pt,285.0957pt);
\node at (57.0643pt,267.0957pt)[below right,color=black]{interrupt};
\end{tikzpicture}

\end{adjustbox}
\end{frame}

\begin{frame}
\frametitle{Coroutine}
\begin{adjustbox}{max totalsize={.9\textwidth}{.7\textheight},center}

% generated by Plantuml 1.2018.00      
\definecolor{plantucolor0000}{RGB}{255,255,255}
\definecolor{plantucolor0001}{RGB}{74,100,132}
\definecolor{plantucolor0002}{RGB}{145,198,255}
\definecolor{plantucolor0003}{RGB}{0,0,0}
\begin{tikzpicture}[yscale=-1
,pstyle0/.style={color=plantucolor0001,fill=white,line width=1.0pt}
,pstyle1/.style={color=plantucolor0001,line width=1.0pt,dash pattern=on 5.0pt off 5.0pt}
,pstyle2/.style={color=black,fill=plantucolor0002,line width=1.5pt}
,pstyle3/.style={color=plantucolor0001,fill=plantucolor0001,line width=1.0pt}
,pstyle4/.style={color=plantucolor0001,line width=1.0pt}
]
\draw[pstyle0] (28.8pt,49.7461pt) rectangle (38.8pt,71.7461pt);
\draw[pstyle0] (28.8pt,102.2246pt) rectangle (38.8pt,132.7031pt);
\draw[pstyle0] (199.704pt,71.7461pt) rectangle (209.704pt,102.2246pt);
\draw[pstyle0] (199.704pt,132.7031pt) rectangle (209.704pt,163.1816pt);
\draw[pstyle1] (33pt,39.7461pt) -- (33pt,181.6602pt);
\draw[pstyle1] (203.9879pt,39.7461pt) -- (203.9879pt,181.6602pt);
\draw[pstyle2] (8pt,3pt) rectangle (55.6pt,34.7461pt);
\node at (15pt,10pt)[below right,color=black]{Main};
\draw[pstyle2] (8pt,180.6602pt) rectangle (55.6pt,212.4063pt);
\node at (15pt,187.6602pt)[below right,color=black]{Main};
\draw[pstyle2] (158.9879pt,3pt) rectangle (246.4201pt,34.7461pt);
\node at (165.9879pt,10pt)[below right,color=black]{Coroutine1};
\draw[pstyle2] (158.9879pt,180.6602pt) rectangle (246.4201pt,212.4063pt);
\node at (165.9879pt,187.6602pt)[below right,color=black]{Coroutine1};
\draw[pstyle0] (28.8pt,49.7461pt) rectangle (38.8pt,71.7461pt);
\draw[pstyle0] (28.8pt,102.2246pt) rectangle (38.8pt,132.7031pt);
\draw[pstyle0] (199.704pt,71.7461pt) rectangle (209.704pt,102.2246pt);
\draw[pstyle0] (199.704pt,132.7031pt) rectangle (209.704pt,163.1816pt);
\draw[pstyle3] (187.704pt,67.7461pt) -- (197.704pt,71.7461pt) -- (187.704pt,75.7461pt) -- (191.704pt,71.7461pt) -- cycle;
\draw[pstyle4] (33.8pt,71.7461pt) -- (193.704pt,71.7461pt);
\node at (40.8pt,53.7461pt)[below right,color=black]{start (context switch)};
\draw[pstyle3] (49.8pt,98.2246pt) -- (39.8pt,102.2246pt) -- (49.8pt,106.2246pt) -- (45.8pt,102.2246pt) -- cycle;
\draw[pstyle4] (43.8pt,102.2246pt) -- (203.704pt,102.2246pt);
\node at (55.8pt,84.2246pt)[below right,color=black]{yield};
\draw[pstyle3] (187.704pt,128.7031pt) -- (197.704pt,132.7031pt) -- (187.704pt,136.7031pt) -- (191.704pt,132.7031pt) -- cycle;
\draw[pstyle4] (33.8pt,132.7031pt) -- (193.704pt,132.7031pt);
\node at (40.8pt,114.7031pt)[below right,color=black]{reenter (context switch)};
\draw[pstyle3] (44.8pt,159.1816pt) -- (34.8pt,163.1816pt) -- (44.8pt,167.1816pt) -- (40.8pt,163.1816pt) -- cycle;
\draw[pstyle4] (38.8pt,163.1816pt) -- (203.704pt,163.1816pt);
\node at (50.8pt,145.1816pt)[below right,color=black]{yield};
\end{tikzpicture}
\end{adjustbox}
\end{frame}

\begin{frame}[fragile]
\frametitle{Thread code example}

\begin{lstlisting}
mutex mtx; 
thread t1, t2;
{
    lock_guard lock(mtx);    
    t1 = [mtx]{
            { lock_guard lock(mtx); f1(); }
            { lock_guard lock(mtx); f2(); }
        };
    t2 = [mtx]{
            { lock_guard lock(mtx); f3(); }
            { lock_guard lock(mtx); f4(); }
        };   
} //could be in any order, just guarantees
  //that f2 is called after f1 and f4 after f3
t1.join(); 
t2.join();
\end{lstlisting}
\end{frame}

\begin{frame}[fragile]
\frametitle{Coroutine code example}

\begin{lstlisting}
coroutine c1 = {f1(); yield(); f2();};
coroutine c2 = {f3(); yield(); f4();};    

c1.enter(); //after yield in c1
c2.enter(); //after yield in c2
c1.reenter(); //reenters after yield
c2.reenter(); //reenters after cield

//always executes f1, f3, f2, f4

\end{lstlisting}
\end{frame}

\begin{frame}
\frametitle{Argument passing \& value returning}

\begin{itemize}
\item<1-> No concurrent access
\item<2-> Deterministic context switch
\item<3-> Pass values in and out through \lstinline{yield}
\end{itemize}
\end{frame}

\begin{frame}[fragile]
\frametitle{Argument passing \& value returning}
\begin{lstlisting}
coroutine<double(int)> cr = 
    [](int in) -> double
    {
         in = yield(in * 0.5);
         in = yield(in * 0.25);
         return in * 1.2;
    };

assert(cr.enter(4) == 2.0);   //4 * 0.5
assert(cr.reenter(2) == 0.5); //2 * 0.025
assert(cr.reenter(3) == 3.6); //3 * 1.2
    
\end{lstlisting}

\end{frame}

\begin{frame}[fragile]
\frametitle{Context}
\begin{itemize}
\item<1-> \lstinline[columns=fixed]{   r13/    SP  } Stack pointer
\item<2-> \lstinline[columns=fixed]{   r14/    LR  } Link Register
\item<3-> \lstinline[columns=fixed]{r4-r11/v1-v8  } Variable registers
\item<4-> \lstinline[columns=fixed]{r0- r3/a1-a4  } Argument/Scratch register
\end{itemize}

\end{frame}

\begin{frame}[fragile]
\frametitle{Stack Pointer}
\begin{itemize}
\item<1-> Points to lowest element on stack
\item<2-> Decremented on function entry / incremented on exit (usually)
\item<3-> Relative validity inside function
\end{itemize}

\begin{block}<4->{}
\begin{lstlisting}
void foo() { }
\end{lstlisting}
\end{block}

\begin{block}<5->{}
\begin{lstlisting}[language=ASM]
  str fp, [sp, #-4]!
  add fp, sp, #0
  mov r0, r0 @ nop
  sub sp, fp, #0
  ldr fp, [sp], #4
  bx lr
\end{lstlisting}
\end{block}

\end{frame}

\begin{frame}[fragile]
\frametitle{Link Register}
\begin{itemize}
\item<1-> Points to the code location of function call
\item<2-> Returning means jump to the location
\item<3-> Pushed on stack for new function call
\end{itemize}

\begin{block}<4->{}
\begin{lstlisting}
void foo() {bar();}
\end{lstlisting}
\end{block}

\begin{block}<5->{}
\begin{lstlisting}[language=ASM]
  stmfd sp!, {fp, lr}
  add fp, sp, #4
  bl bar()
  mov r0, r0 @ nop
  sub sp, fp, #4
  ldmfd sp!, {fp, lr}
  bx lr
\end{lstlisting}
\end{block}

\end{frame}

\begin{frame}
\frametitle{Variable registers}
\begin{itemize}
\item<1-> Load/Store-Architecture
\item<2-> Values must be loaded into variable registers for operations
\item<3-> Must be persistent after subroutine calls
\item<4-> When used, old values get pushed on the stack
\end{itemize}

\end{frame}

\begin{frame}[fragile]
\frametitle{Argument Register}
\begin{itemize}
\item<1-> Only valid in local context, not persistent
\item<2-> Used to pass arguments in and return
\item<3-> Large values store a reference
\end{itemize}

\begin{block}<4->{}
\begin{lstlisting}
extern int in_, out;
int foo(int in) {in_ = in; return out;}
\end{lstlisting}
\end{block}

\begin{block}<5->{}
\begin{lstlisting}[language=ASM]
  ldr r2, .L2
  ldr r3, .L2+4
  str r0, [r2]
  ldr r0, [r3]
  bx lr
\end{lstlisting}
\end{block}

\end{frame}


\end{document}
