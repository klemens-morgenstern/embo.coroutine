% This text is proprietary.
% It's a part of presentation made by myself.
% It may not used commercial.
% The noncommercial use such as private and study is 

\documentclass{beamer}
\usepackage{xcolor}
\usepackage{tikz}
\usepackage{aeguill}
\usepackage{adjustbox}
\usepackage{listings}
\usepackage[style=british]{csquotes}
\usepackage{pgfpages}

\def\tikzmark#1{\tikz[remember picture,overlay]\node[yshift=2pt](#1){};}

\usetikzlibrary{arrows.meta}

%\setbeamertemplate{note page}[plain]
%\setbeameroption{show notes on second screen=right}
\definecolor{darkgreen}{rgb}{0.33, 0.42, 0.18}
\usetheme{metropolis}
%\usecolortheme{dolphin}
\setbeamercolor{block body}{bg=gray!30,fg=black}
\setbeamercolor{block title}{bg=black!40,fg=black}

\begin{document}

\lstset{language=C++,
                basicstyle=\ttfamily,
                keywordstyle=\color{blue}\ttfamily,
                stringstyle=\color{red}\ttfamily,
                commentstyle=\color{darkgreen}\ttfamily,
                numbers=left,               % Ort der Zeilennummern
                stepnumber=1, 
                %numberstyle=\small,
                escapeinside={(*@}{@*)},
                xleftmargin=2.pt
}

\lstdefinelanguage{ASM}{
    morekeywords={str, add, move, sub, ldr, bx, mov,
    stmfd, nop, ldmfd bl, push, pop, bl, pop, cmp, beq, ldmfd, mla},
    sensitive=false, % keywords are not case-sensitive
    morecomment=[l]{@}, % l is for line comment
    morecomment=[s]{/*}{*/}, % s is for start and end delimiter
    morestring=[b]" % defines that strings are enclosed in double quotes
} % 

\title{Developing high-performance Coroutines for ARMs}
\author{Klemens Morgenstern}
\date{10.03.2018}

\frame{\titlepage}

\begin{frame}
\frametitle{Introduction}
\begin{itemize}
\item Electrical engineer by training
\item C++ developer by passion
\item boost contributor
\item Independent contractor
\item Open for consulting
\end{itemize}
\end{frame}

\begin{frame}
\frametitle{Motivation}
\begin{itemize}
\item<1-> Coroutines are awesome
\item<2-> Low overhead alternative to threads 
\item<3-> Do not require an OS / Scheduler
\item<4-> More readable alternative to state-machines
\item<5-> Not well enough known 
\end{itemize}
\end{frame}


\begin{frame}
\frametitle{Overview}
\begin{enumerate}
\item<1-> Introduction to coroutines
\item<2-> Implementation
\item<3-> Performance comparisons
\end{enumerate}
\end{frame}

\note[itemize]{
\item Awesome, not well known enough
\item Light-weight alternative to threads
\item Alternative to Statemachines
\item Higher performance than often assumed
\item Perfect for none-eabi, no thread-scheduler needed.
}

\begin{frame}
\frametitle{Definition}

\begin{itemize}
\item<1-> Function with it's own stack
\item<2-> can interrupt it's own execution
\item<3-> and be resumed later
\end{itemize}
  
\end{frame}

\begin{frame}[fragile]
\frametitle{Subroutine}
\begin{columns}
\begin{column}{0.5\textwidth}
\begin{block}{\vspace*{-3.2ex}}
\begin{lstlisting}
int main()
{
  subroutine();
  return 0;
}

void subroutine()
{
  some_work();
  more_work();
}
\end{lstlisting}
\end{block}
\end{column}
\begin{column}{0.5\textwidth}
\begin{adjustbox}{max totalsize={.9\textwidth}{.7\textheight},center}
\input{uml/subroutine.latex}
\end{adjustbox}
\end{column}
\end{columns}
\end{frame}

\begin{frame}[fragile]
\frametitle{Coroutine}
\begin{columns}
\begin{column}{0.65\textwidth}
\vspace*{-3ex}
\begin{block}{\vspace*{-3.2ex}}
\begin{lstlisting}
int main()
{
  coroutine cr{&cr_impl};
  cr.spawn();
  //...main does something
  cr.reenter();
  return 0;
}

void cr_impl(yield_t &yield)
{
  some_work();
  yield();
  more_work();
}
\end{lstlisting}
\end{block}
\end{column}
\begin{column}{0.35\textwidth}
\begin{adjustbox}{max totalsize={.99\textwidth}{.99\textheight},center}

% generated by Plantuml 1.2018.00      
\definecolor{plantucolor0000}{RGB}{255,255,255}
\definecolor{plantucolor0001}{RGB}{74,100,132}
\definecolor{plantucolor0002}{RGB}{145,198,255}
\definecolor{plantucolor0003}{RGB}{0,0,0}
\begin{tikzpicture}[yscale=-1
,pstyle0/.style={color=plantucolor0001,fill=white,line width=1.0pt}
,pstyle1/.style={color=plantucolor0001,line width=1.0pt,dash pattern=on 5.0pt off 5.0pt}
,pstyle2/.style={color=black,fill=plantucolor0002,line width=1.5pt}
,pstyle3/.style={color=plantucolor0001,fill=plantucolor0001,line width=1.0pt}
,pstyle4/.style={color=plantucolor0001,line width=1.0pt}
]
\draw[pstyle0] (28.8pt,49.7461pt) rectangle (38.8pt,71.7461pt);
\draw[pstyle0] (28.8pt,102.2246pt) rectangle (38.8pt,132.7031pt);
\draw[pstyle0] (199.704pt,71.7461pt) rectangle (209.704pt,102.2246pt);
\draw[pstyle0] (199.704pt,132.7031pt) rectangle (209.704pt,163.1816pt);
\draw[pstyle1] (33pt,39.7461pt) -- (33pt,181.6602pt);
\draw[pstyle1] (203.9879pt,39.7461pt) -- (203.9879pt,181.6602pt);
\draw[pstyle2] (8pt,3pt) rectangle (55.6pt,34.7461pt);
\node at (15pt,10pt)[below right,color=black]{Main};
\draw[pstyle2] (8pt,180.6602pt) rectangle (55.6pt,212.4063pt);
\node at (15pt,187.6602pt)[below right,color=black]{Main};
\draw[pstyle2] (158.9879pt,3pt) rectangle (246.4201pt,34.7461pt);
\node at (165.9879pt,10pt)[below right,color=black]{Coroutine1};
\draw[pstyle2] (158.9879pt,180.6602pt) rectangle (246.4201pt,212.4063pt);
\node at (165.9879pt,187.6602pt)[below right,color=black]{Coroutine1};
\draw[pstyle0] (28.8pt,49.7461pt) rectangle (38.8pt,71.7461pt);
\draw[pstyle0] (28.8pt,102.2246pt) rectangle (38.8pt,132.7031pt);
\draw[pstyle0] (199.704pt,71.7461pt) rectangle (209.704pt,102.2246pt);
\draw[pstyle0] (199.704pt,132.7031pt) rectangle (209.704pt,163.1816pt);
\draw[pstyle3] (187.704pt,67.7461pt) -- (197.704pt,71.7461pt) -- (187.704pt,75.7461pt) -- (191.704pt,71.7461pt) -- cycle;
\draw[pstyle4] (33.8pt,71.7461pt) -- (193.704pt,71.7461pt);
\node at (40.8pt,53.7461pt)[below right,color=black]{start (context switch)};
\draw[pstyle3] (49.8pt,98.2246pt) -- (39.8pt,102.2246pt) -- (49.8pt,106.2246pt) -- (45.8pt,102.2246pt) -- cycle;
\draw[pstyle4] (43.8pt,102.2246pt) -- (203.704pt,102.2246pt);
\node at (55.8pt,84.2246pt)[below right,color=black]{yield};
\draw[pstyle3] (187.704pt,128.7031pt) -- (197.704pt,132.7031pt) -- (187.704pt,136.7031pt) -- (191.704pt,132.7031pt) -- cycle;
\draw[pstyle4] (33.8pt,132.7031pt) -- (193.704pt,132.7031pt);
\node at (40.8pt,114.7031pt)[below right,color=black]{reenter (context switch)};
\draw[pstyle3] (44.8pt,159.1816pt) -- (34.8pt,163.1816pt) -- (44.8pt,167.1816pt) -- (40.8pt,163.1816pt) -- cycle;
\draw[pstyle4] (38.8pt,163.1816pt) -- (203.704pt,163.1816pt);
\node at (50.8pt,145.1816pt)[below right,color=black]{yield};
\end{tikzpicture}
\end{adjustbox}
\end{column}
\end{columns}
\end{frame}

\begin{frame}[fragile]
\frametitle{Coroutine concurrency}
\begin{columns}
\begin{column}{0.60\textwidth}
\vspace*{-3ex}
\begin{block}{\vspace*{-3.2ex}}

\begin{lstlisting}[basicstyle=\small]
int main()
{
  coroutine cr1{&cr_impl};
  coroutine cr2{&cr_impl};
  cr1.spawn();
  cr2.spawn();
  cr1.reenter();
  cr2.reenter();
  return 0;
}
void cr_impl(yield_t & yield)
{
  some_work();
  yield();
  more_work();
}
\end{lstlisting}
\end{block}
\end{column}
\begin{column}{0.4\textwidth}

\begin{adjustbox}{max totalsize={.98\textwidth}{.98\textheight},center}
\input{uml/coroutine2.latex}
\end{adjustbox}
\end{column}
\end{columns}
\end{frame}

\note[itemize]
{
   \item Pseudo-concurrency
   \item No race conditions		
}

\begin{frame}[fragile]
\frametitle{Value yield example}
\vspace*{-3ex}
\begin{block}{\vspace*{-3.2ex}}
\begin{lstlisting}[basicstyle=\small]
int cr_impl(yield_t & yield) 
{
  yield(1);
  yield(2);
  return 4;  
}

int main() 
{
  coroutine<int()> cr{&cr_impl};
  
  assert(cr.spawn() == 1);
  assert(cr.reenter() == 2);
  assert(cr.reenter() == 4);
  return cr.done() ? 0 : 1;
}
\end{lstlisting}

\end{block}
\end{frame}

\begin{frame}[fragile]
\frametitle{Argument passing example}
\vspace*{-3ex}
\begin{block}{\vspace*{-3.2ex}}
\begin{lstlisting}[basicstyle=\small]
int cr_impl(yield_t & yield, bool) 
{
  int value = 1;
  while(yield(value))
    value <<= 1;
  return 0;
}

int main() 
{
  coroutine<int(bool)> cr{&cr_impl};
  
  assert(cr.spawn(true) == 1);
  assert(cr.reenter(true) == 2);
  assert(cr.reenter(true) == 4);
  return cr.reenter(false);
}
\end{lstlisting}
\end{block}
\end{frame}

\begin{frame}
\frametitle{Properties}
\begin{itemize}
\item<1-> Deterministic, no real concurrency
\item<2-> Pass values out on yield in on reenter
\item<3-> May not be moved in memory
\item<4-> Can be stored (e.g. for deep sleep)
\item<5-> Only constraint is the stack-size
\item<6-> Could be used as callback
\end{itemize}
\end{frame}


%%%%%%%%%%%%%%%%%%%%%%%%%%%%%%%%%%%%%% 2 Implementation %%%%%%%%%%%%%%%%%%%%%%%%%%

\begin{frame}[fragile]
\frametitle{Context switch}
\begin{columns}
\begin{column}{0.49\textwidth}
\begin{block}{Main Context}
\begin{lstlisting}[basicstyle=\small]
int main()
{
  coroutine cr;
  cr.spawn()
  /* make_context(*@\tikzmark{A}@*)
   *
   * 
   *return from spawn*/;(*@\tikzmark{D}@*)
  cr.reenter()
  /* switch_context (*@\tikzmark{E}@*)
   *
   *return from reenter*/;(*@\tikzmark{H}@*)
  return 0;
}
\end{lstlisting}
\end{block}
\end{column}
\begin{column}{0.49\textwidth}
\begin{block}{Coroutine Context}
\begin{lstlisting}[basicstyle=\small, numbers=right]




(*@\tikzmark{B}@*)void cr_impl()
{
  yield()
  (*@\tikzmark{C}@*)/* switch_context
   *
   (*@\tikzmark{F}@*)*return from yield*/;
}
(*@\tikzmark{G}@*)

(*@@*)  
\end{lstlisting}
\end{block}
\end{column}
\end{columns}
\begin{tikzpicture}[remember picture, overlay]
\pause{}

\draw[->, red](A) to (B);
\draw[->, red](C) to (D);
\draw[->, red](E) to (F);
\draw[->, red](G) to (H);
\end{tikzpicture}

\end{frame}


\begin{frame}[fragile]
\frametitle{Context switch}
\begin{adjustbox}{max totalsize={.9\textwidth}{.9\textheight},center}
\input{uml/context_switch.latex}
\end{adjustbox}

\end{frame}

\begin{frame}[fragile]
\frametitle{Stack Pointer \lstinline{SP}}
\begin{itemize}
\item<1-> Points to lowest element on stack
\item<2-> Decremented on function entry / incremented on exit (usually)
\end{itemize}

\begin{block}<4->{C++}
\begin{lstlisting}
void foo() { }
\end{lstlisting}
\end{block}

\begin{block}<5->{ASM (by gcc)}
\begin{lstlisting}[language=ASM]
  str fp, [sp, #-4]!
  add fp, sp, #0
  sub sp, fp, #0
  mov a1, a1 @ nop
  sub sp, fp, #0
  ldr fp, [sp], #4
  bx lr
\end{lstlisting}
\end{block}

\end{frame}

\begin{frame}[fragile]
\frametitle{Link Register \lstinline{LR}}
\begin{itemize}
\item<1-> Points to the code location of function call
\item<2-> Returning means jump to the location
\item<3-> Pushed on stack for new function call
\end{itemize}

\begin{block}<4->{C++}
\begin{lstlisting}
void foo() {bar();}
\end{lstlisting}
\end{block}

\begin{block}<5->{ASM (by gcc)}
\begin{lstlisting}[language=ASM]
  push {fp, lr}
  add fp, sp, #4
  bl bar()
  mov r0, r0 @ nop
  sub sp, fp, #4
  pop {fp, lr}
  bx lr
\end{lstlisting}
\end{block}

\end{frame}

\begin{frame}[fragile]
\frametitle{Argument Register \lstinline{a1-a4}}
\begin{itemize}
\item<1-> Used to pass arguments in and return
\item<2-> Large values store a reference
\item<3-> \lstinline{a1} stores the return value
\item<4-> Only valid in local context, not persistent
\item<5-> \lstinline{ip} is an additional scratch register
\end{itemize}

\begin{block}<5->{C++}
\begin{lstlisting}
int foo(int x, int y) { return x+y; }
\end{lstlisting}
\end{block}

\begin{block}<6->{ASM (by gcc,-O3)}
\begin{lstlisting}[language=ASM]
  add a1, a1, a2
  bx lr
\end{lstlisting}
\end{block}

\end{frame}

\begin{frame}[fragile]
\frametitle{Variable registers \lstinline{v1-v8}}
\begin{itemize}
\item<1-> Load/Store-Architecture
\item<2-> Values must be loaded into variable registers for operations
\item<3-> Must be persistent after subroutine calls
\item<4-> When used, old values get pushed on the stack
\end{itemize}

\begin{columns}
\begin{column}{0.55\textwidth}
\begin{block}<5->{C++}
\begin{lstlisting}[basicstyle=\small]
void bar(int&, int&, 
         int&, int&);
void foo(int &x, int &y, 
         int &z, int &i)
{ 
    x = i+y+z; 
    bar(x,y,z,i); 
}\end{lstlisting}
\end{block}
\end{column}
\begin{column}{0.45\textwidth}
\begin{block}<6->{ASM (by gcc)}
\begin{lstlisting}[language=ASM,basicstyle=\small]
  push {v1, v2, v3, lr}
  ldr v2, [a4]
  ldr v1, [a2]
  ldr lr, [a3]
  mla ip, v1, v2, lr
  str ip, [a1]
  bl bar
  pop {v1, v2, v3, lr}
  bx lr
\end{lstlisting}
\end{block}
\end{column}
\end{columns}

\end{frame}

\begin{frame}[fragile]
\frametitle{Context}
\begin{itemize}
\item<1-> \lstinline[columns=fixed]{r0- r3/a1-a4 } Argument/Scratch register
\item<2-> \lstinline[columns=fixed]{r4-r11/v1-v8 } Variable registers
\item<3-> \lstinline[columns=fixed]{   r11/   fp }\textit{Frame pointer on gcc}
\item<4-> \lstinline[columns=fixed]{   r12/   ip } Intra procedure scratch 
\item<5-> \lstinline[columns=fixed]{   r13/   SP } Stack pointer
\item<6-> \lstinline[columns=fixed]{   r14/   LR } Link Register
register
\end{itemize}
\end{frame}

\begin{frame}[fragile]
\frametitle{Creating a coroutine}
\vspace*{-3ex}

\begin{block}{\vspace*{-3.2ex}}
\begin{lstlisting}[basicstyle=\small]
struct coroutine
{
  void * stack_pointer; //valid
  template<typename Function>
  Return spawn(Function && func)
  {
      return make_context(this, &func, &executor);
  }
  static void executor(coroutine * const this_, 
                       Function *func_p)
  {
    Function func = *func_p;
    Return val = func({this_});
    done = true;
    switch_context(val, this_);
  };  
};
\end{lstlisting}
\end{block}
\end{frame}


\begin{frame}[fragile]
\frametitle{Make context}

\begin{block}{\vspace*{-3.2ex}}
\begin{lstlisting}[language=ASM, basicstyle=\small]
make_context_0:
    @_make_context_0(cr* const, void *, void * )
    @                coroutine, func,   executor
    @executor: (cr * const, void *)
    @           coroutine,  func
    push {v1-v8, lr} @push  the link register
    mov v1, sp   @move the stack pointer to v1
    ldr sp, [a1] @set the stack pointer
    str v1, [a1] @store the old stack pointer

    bx a3 @call the function

\end{lstlisting}
\end{block}
\end{frame}

\begin{frame}[fragile]
\frametitle{Context switch}
\vspace*{-2ex}
\begin{block}<1->{\vspace*{-3.2ex}}
\begin{lstlisting}
void switch_context_0(void*& sp);
void yield  (void*& p){switch_context_0(p);}
void reenter(void*& p){switch_context_0(p);}

\end{lstlisting}
\end{block}
\vspace*{-2ex}
\begin{block}<2->{\vspace*{-3.2ex}}
\begin{lstlisting}[language=ASM]
switch_context_0:
  push {v1-v8, lr} @8
  
  mov v1, sp   @1
  ldr sp, [a1] @2
  str v1, [a1] @2	

  pop {v1-v8, lr} @8
  bx lr @1  --> overall 22
\end{lstlisting}
\end{block}
\end{frame}

\begin{frame}[fragile]
\frametitle{Context switch \lstinline{uint32_t(uint64_t)}}
\begin{block}{\vspace*{-3.2ex}}
\begin{lstlisting}[basicstyle=\small]
uint64_t switch_context_1(uint32_t, void*& sp);
uint32_t switch_context_2(uint64_t, void*& sp);
uint64_t yield  (uint32_t i, void*& p)
{
  return switch_context_1(i, sp);
}
uint32_t reenter(uint64_t i, void*& p)
{
  return switch_context_2(i, sp);
}
\end{lstlisting}
\end{block}
\end{frame}

\begin{frame}[fragile]
\frametitle{Context switch \lstinline{yield(uint32_t)}}
%\begin{block}{}
%\begin{lstlisting}[basicstyle=\small]
%uint64_t yield  (uint32_t i, void*& p);
%\end{lstlisting}
%\end{block}
\begin{block}{\vspace*{-3.2ex}}
\begin{lstlisting}[language=ASM, basicstyle=\small]
switch_context_1:
    @yield(uint32_t, void*) -> uint64_t
    @      a1      , a2
    push {v1-v8, lr} 

    mov v1, sp
    ldr sp, [a2]
    str v1, [a2]

    pop {v1-v8, lr}

    @reenter(uint64_t, void*) -> uint32_t
    @return value is in a1
    bx lr
\end{lstlisting}
\end{block}
\end{frame}

\begin{frame}[fragile]
\frametitle{Context switch \lstinline{reenter(uint64_t)}}
%\begin{block}{}
%\begin{lstlisting}[basicstyle=\small]
%uint32_t reenter  (uint64_t i, void*& p);
%\end{lstlisting}
%\end{block}
\begin{block}{\vspace*{-3.2ex}}
\begin{lstlisting}[language=ASM, basicstyle=\small]
switch_context_2:
    @reenter(uint64_t, void*) -> uint32	_t
    @        a1-a2   , a3
    push {v1-v8, lr}

    mov v1, sp
    ldr sp, [a3]
    str v1, [a3]

    pop {v1-v8, lr}

    @yield(uint32_t, void*) -> uint64_t
    @return value is in a1-a2
    bx lr
\end{lstlisting}
\end{block}
\end{frame}



%%%%%%%%%%%%%%%%%%%%%%%%%%%%%%%%%%%%%%%%%%%% THINGY %%%%%%%%%%%%%%%%%%%%%%%%%%%%%%%%
\begin{frame}[fragile]
\frametitle{Statemachine example}
\vspace*{-3ex}
\begin{block}{\vspace*{-3.2ex}}
\begin{lstlisting}[basicstyle=\small]
struct statemachine
{
  int state = 0;
  bool done = false;
  int operator()()
  {
    switch (state)
    {
      case 0:        
        state = 1; return f();
      case 1:
        state = 2; return g(); 
      case 2:
        done = true; return h();
    }
  }
};
\end{lstlisting}
\end{block}
\end{frame}

\begin{frame}[fragile]
\frametitle{Statemachine example}
\begin{block}{\vspace*{-3.2ex}}
\begin{lstlisting}[basicstyle=\small]]
int coroutine(embo::yield_t<int()>& yield)
{
    yield(f());
    yield(g());
    return h();
}
\end{lstlisting}
\end{block}
\end{frame}

\begin{frame}[fragile]
\frametitle{Statemachine Assembly}
\vspace*{-3ex}
\begin{block}{\vspace*{-3.2ex}}
\begin{lstlisting}[language=ASM, basicstyle=\small]
statemachine::operator()():
  ldr a4, [a1]  @2
  push {v1, lr} @1
  cmp a4, #0    @1
  beq .L10      @1 
  cmp a4, #1
  beq .L3
  cmp a4, #2
  beq .L4
  pop {r4, lr}
  bx lr
.L10:
  mov a4, #1    @1
  str a4, [a1]  @2
  bl f()        @1
  pop {v1, lr}  @1
  bx lr         @1 -> 11 for f, 12 for g, 13 for h
\end{lstlisting}
\end{block}
\end{frame}

\begin{frame}[fragile]
\frametitle{Coroutine Assembly}
\begin{block}{\vspace*{-3.2ex}}
\begin{lstlisting}[language=ASM, basicstyle=\small]
coroutine(yield_t&): @14
  push {v1, lr} @1
  mov v1, a1	@1  
  bl f()        @1
  mov a2, v1    @1
  bl switch_context_1(int, void*) @1+22; 22 reentry
  bl g() @1
  mov a2, v1 @1
  bl switch_context_1(int, void*) @1+22
  bl h() @1
  pop {v1, lr} @1
  bx lr @1 -> 41 for f, 47 g, 47 for h
\end{lstlisting}
\end{block}
\end{frame}

\begin{frame}[fragile]
\frametitle{Statemachine stack example}
\vspace*{-3ex}

\begin{block}{\vspace*{-3.2ex}}
\begin{lstlisting}[basicstyle=\small]
struct statemachine {
  int state = 0;
  bool use_wchar = false;
  array<char, 128> buffer;
  array<wchar_t, 128> wbuffer;
  void operator()() {
    switch (state) {
      case 0: state = 1; use_wchar = get_mode(); 
      break;
      case 1: 
        if (use_wchar) {state = 2; read(wbuffer);}
        else           {state = 3; read( buffer);}
        return;
      case 2: handle(wbuffer); return; 
      case 3: handle( buffer); return;
} } };
\end{lstlisting}
\end{block}
\end{frame}

\begin{frame}[fragile]
\frametitle{Coroutine stack example}
\vspace*{-3ex}

\begin{block}{\vspace*{-3.2ex}}
\begin{lstlisting}[basicstyle=\small]
void cr(embo::yield_t<void()>& yield)
{
  bool use_wchar = get_mode();
  yield();
  if (use_wchar)
  {
    array<wchar_t, 128> wbuffer;
    read(wbuffer); yield();
    handle(wbuffer);
  }
  else
  {
    array<char, 128> buffer;
    read(buffer); yield();
    handle(buffer);
  }
}
\end{lstlisting}
\end{block}
\end{frame}

\begin{frame}[fragile]
\frametitle{Coroutine stack template example}
\vspace*{-3ex}
\begin{block}{\vspace*{-3.2ex}}
\begin{lstlisting}[basicstyle=\small]
template<typename Char>
void cr_impl(embo::yield_t<void()> &yield)
{
  array<Char, 128> buffer;
  read(buffer); yield();
  handle(buffer);
}

void cr(embo::yield_t<void()>& yield)
{
  bool use_wchar = get_mode();
  yield();
  if (use_wchar)
    cr_impl<wchar_t>(yield);
  else
    cr_impl<char>(yield);
}
\end{lstlisting}
\end{block}
\end{frame}

\begin{frame}
\frametitle{Summary}
\begin{itemize}
\item<1-> www.github.com/klemens-morgenstern/embo.coroutine
\item<2-> Any questions?
\end{itemize}

\end{frame}

\end{document}
